\chapter{Descrizione tecnologie usate}
	Nel seguente capitolo illustererò le varie metodologie e tecnologie utilizzate, con attenzione al frmework e le api.
	\section{Ruby e Ruby on Rails} 
	Ruby è un linguaggio di programmazione nato nel 1993 che, pur essendo ad oggetti, presenta alcune caratteristiche tipiche dei paradigmi imperativo e funzionale.
	
	Il paradigma ad oggetti di \textbf{Ruby} è puro, ossia ogni componente del linguaggio, dalle costanti numeriche alle classi, è un oggetto, e come tale può possedere metodi; a differenza dei linguaggi come C++ e derivati, tuttavia, gli oggetti in Ruby sono qualcosa di molto più dinamico, in quanto è possibile aggiungere o modificare metodi a run-time. Il tipo di un oggetto, perciò, non è definito tanto dalla classe che lo ha istanziato, quanto dall'insieme dei metodi che possiede, o dei messaggi a cui sa rispondere.
	
	In Ruby, dunque, è fondamentale il \textit{duck typing} (dall'inglese if it looks like a duck, and quacks like a duck, it must be a duck), ovvero il principio secondo il quale il comportamento di una funzione sui suoi argomenti non deve essere determinato dal tipo di questi (come accade in C++ e altri linguaggi staticamente tipizzati), bensì da quali messaggi essi sono in grado di gestire.
	
	Un'altra caratteristica fondamentale di Ruby è costituita dai cosiddetti blocchi, che sono sostanzialmente delle chiusure (ovvero funzioni dotate di ambiente), e che consentono di sostituire i cicli espliciti, frequenti nei linguaggi a basso livello, con l'utilizzo di iteratori, nascondendo così al chiamante i meccanismi interni del ciclo in questione.
	\\\\
	Ruby è il linguaggio del framework open source \textbf{Ruby on rails}, per applicazioni web. La sua architettura è una \textbf{MVC} (Model - view - controller). I suoi obiettivi sono la semplicità e la possibilità di 
	Ruby on rails è distribuito attraverso \textbf{RubyGems}, l'unico canale ufficiale di distribuzione per librerie ed applicazioni Ruby.
	
	I principi guida di Ruby on Rails comprendono \textit{don't repeat yourself} e \textit{convention over configuration}.
	
	\textbf{Don't repeat yourself} significa che le definizioni devono essere poste una volta soltanto. Poiché Ruby On Rails è un framework \textit{full-stack}, i componenti sono integrati in modo tale che i collegamenti fra di essi non devono essere impostati manualmente. Ad esempio in Active Record, le definizioni delle classi non devono specificare i nomi delle colonne; Ruby può estrarli direttamente dal database, dunque riportarli anche nel programma sarebbe ridondante.
	
	\textbf{Convention over configuration} significa che il programmatore ha bisogno di metter mano alla configurazione soltanto per ciò che differisce dalle convenzioni. Ad esempio, se un modello è costituito dalla classe Post, la corrispondente tabella nel database deve chiamarsi posts, o altrimenti deve essere specificata manualmente.
	
	\section{Google calendar Api}
	E' stato deciso a monte che il progetto fosse di tipo gestionale, come potevamo non includere il calendario di google?
	
	\textbf{Google calendar} è un  sistema calendari concepito da Google ed offre la possibilità creare più celandari, di condividerli e importarli da altri servizi, il che, calza a pennello con l'obbiettivo di \textit{Inline}, creare eventi e ottimizzare la gestione del tempo.

	L'implementazione è fatta utilizzando la gemma \textit{googleauth}, si avrebbe potuto scegliere anche una soluzione di basso livello scrivendo le varie chiamate \textbf{HTTP} from scratch, ma si è preferito utilizzare una libreria che con poche chiamate fa il suo lavoro.
	
	L'applicazione dispone di un account di servizio in modo che ogni presente sul sito lo sia anche sul calendario. Diverso invece è per gli utenti: l'evento non viene aggiunto automaticamente al proprio calendario, ma è necessario cliccare sul link ipertestuale inserito nella room. Per semplicità si è evitato di incorporare evento ricorsivi nel calendario degli utenti.
	
	\section{Google e facebook oauth}
	\textit{Facebook} e \textit{Google} mettono a disposizione le proprie API per poter ottenere le autorizzazioni per effettuare sign-in o sign-up. E' neccessario essere iscritto ad entrambi i siti e ottenere un account sviluppatore.
	
	Dal 2018 Facebook ha introdotto il vincolo  per l'utilizzo dell'api. Ogni applicazione che utilizzi tali servizi neccessita di un certificato \textit{SSL}. Per superare questo problema in fase di development abbiamo dovuto utilizzare un nostro certificato. E' doveroso bootare il server con il comando:
	\begin{lstlisting}
		rails s -b 'ssl://127.0.0.1:3000?key=localhost.key&cert=localhost.crt
	\end{lstlisting}
	
	\section{Mapbox}
	Mapbox è uno dei più grandi fornitori di mappe online, abbiamo preferito l'utilizzo di questa API rispetto a \textit{Google maps} poichè già molto servizi erano dipendenti dal grande colosso americano. In caso di guasto ci sarebbero stati molti problemi. L'api è stata implementata solo a livello di front-end, tramite librerie \textit{JQuery} e \textit{Javascript}. Per ottenere la chiave per usare questo applicativo abbiamo bisogno di un account sviluppatore nel sito di mapbox.
	
	Nel sito è presente la mappa nella dashboard, dove presenta le room vicine alla posizione ricercata o a quella attuale, ottenuta tramite GPS, nella creazione della room che però, a differenza della precedente, è nascosta all'utente finale. Questo perchè è necessario che sia presente per far funzionare il ricercatore di località Mapbox. Ed infine, è presente nelle stanze che hanno una posizione, nella colonna sinistra. 
	
	\section{Mailboxer}
	Mailboxer è una gemma Rails che fa parte del framework \textit{social stream} per la creazione di social network. È un sistema di messaggistica generico che permette di rendere messaggiabile qualsiasi modello, dotandolo di alcuni strumenti versatili. Con Mailboxer, puoi creare conversazioni con uno o più destinatari e notificare via e-mail. È anche possibile inviare messaggi tra diversi modelli e aggiungere allegati. In poche parole, è lo strumento che ci ha permesso di creare i messaggi privati all'interno della piattaforma.
	
	\section{Bootstrap}
	Bootstrap è una raccolta di librearia per la creazione di siti e applicazione per il web. E' stata sviluppata presso \textit{Twitter} come un framework che uniformasse i vari componenti che ne realizzavano l'interfaccia web, dato che la presenza di diverse libreria aveva portato ad incoerenze ed eleveti oneri di manutenzione.
	
	Continene modelli basati su \textit{CSS} e \textit{HTML} per l'interfaccia grafica, contenendo quindi, bottoni, moduli e pulsanti di navigazioni. E' stato incluso con la gemma \textit{bootstrap} e poi chiamato tramite \textit{SASS}.
	
	\section{Ice cube}
	Ice cube è una gemma di rails che gestire eventi ripetuti facilmente. L'API è modellata sugli eventi iCalendar in una piacevole sinstassi di Ruby. Il potere sta nella capacità di specificare regole multiple e fare in modo che Ice cube capisce rapidamente l'evento schedulato cada in una determinata data o in quali orari si verifica.
	
	E' stata presentata in un tutorial di \textit{Go rails}, un famoso canale che porta notizie sul framework in questione, dove ne mostrava l'utilizzo.
	
	\section{Strumenti minori}
	Altri strumenti che non \textit{meritano} una sezione a parte, poichè non hanno avuto un ruolo fondamentale nello sviluppo del sito sono: 
	\begin{itemize}	
		\item \textbf{Paperclip}, una gemma utilizzata per ritagliare gli avatar delle Room.
		\item \textbf{Bcrypt}, una gemma utilizzata per encryptare le password degli utenti in fase di creazione
		\item \textbf{Devise invitable}, fa parte del pacchetto devise e serve per invitare persone via mail
		\item \textbf{Friendly ID}, genera un hash e lo inserisce al posto dell'ID utente e delle rooms, per questioni di sicurezza.
	\end{itemize}

	\section{Metodologie usate}
	Nello sviluppo dell'applicazione sono stati utilizzati diversi strumenti. Il principale è stato \textit{Git-Hub}, un famoso sito per sviluppatori comprato di recente dalla \textit{Microsoft}. Ci ha permesso di essere sempre aggiornato con il lavoro degli altri componento del team. 
	
	Io, personalmente, ho utilizzato Visual studio code della \textit{Microsoft}, come editor di testo. Sebbene sia uno strumento pesanto, mi ha aiutato molto alla stesura di un codice pulito e facile da legge.
	
	E' stato utilizzato uno sito esterno, \textit{Taiga}, il tracking delle user stories e per gli sprint finali. E' stato lo strumento che ci ha permesso di dividere il lavoro per quattro persone in modo equivalente, sia per difficoltà che lungheza.