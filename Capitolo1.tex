\chapter{Introduzione}
	\section{Nascita dell'idea}
	La tesi nasce per l'esame di \textbf{Architetture software e sicurezza informatica}, un laboratorio svolto dal Prof. \textbf{Leonardo Querzoni}. Con una rapida visione sul mercato di software abbiamo notato,io ed il mio team, la mancanza di un sistema per gestire delle code generiche.\\\\
	Confrontandoci con le aziende nel campo IT abbiamo notato l'uso assiduo di \textsl{Google calendar}, prodotto della grande multinazionale \textsl{Google}. Nonostante la grandissima qualità del prodotto, non è possibile organizzare code, ma soltanto eventi a cui è possibile partecipare o promemoria.E' qui che nasce l'opportunità e la voglia di sviluppare qualcosa che ancora non esiste.\\\\
	L'utente dopo essersi registrato può partecipare alla collettività di Inline, creando o partecipando a stanze in  modo facile ed intuitivo. Le room posso essere private o pubbliche, a discrezione del creatore. Se l'utente mantiene la promessa di partecipazione guadagna un punto, visibile nella classifica globale del sito.
	
	
	\section{Organizzazione della tesi}
	Il documento attraverserà la fase di collezionamento dei requisiti, sviluppo e testing in modalità verbosa e, tavolta, accompagnata da grafici ER o User stories.\\
	Ci soffermeremo sulle APIs scelte, menzionando i motivi di prevalenza rispetto ad altre ed i processi di inserimento all'interno del progetto.
	Menzioneremo tutte le user stories del proggetto con mockup di riferimento, e la suddivisione dei compiti. Tutti i test sviluppati verranno affrontati.