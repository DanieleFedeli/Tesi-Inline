\chapter{Test e validazione del software}

	\section{Installazione}
	Inline è stato pensato e costruito sul web framework \textbf{Ruby on Rails}, pertanto, la compatibilità con i sistemi dipende esclusivamente dal framework. Esso è disponibile per i più popolari come Linux, Mac OS e Windows; E' possibile utilizzare terze parti o il terminale per installarlo.
	
	Abbiamo bisogno di \textbf{Ruby} installato nel sistema operativo, alla versione \textbf{2.5.5} come specificato nel Gemfile del software. Successivamente si necessita di alcune librerie fondamentali per lo sviluppatore installabili con il comando
	\begin{lstlisting}
		sudo apt-get update && sudo apt-get install build-essential -y
	\end{lstlisting}
	
	Successivamente, dobbiamo scaricare la cartella che contiene il software. E' necessario avere git installato nel proprio sistema operativo, se non lo è al momento, bisogna scrivere nel terminale:
	\begin{lstlisting}
		sudo apt-get update && sudo apt-get install git -y
	\end{lstlisting}
	
	Dopociò, bisogna effettuare il clone della cartella e abbiamo due alternative: Scaricarla da git attraverso il link: https://github.com/Mattens15/Inline-LASESI, oppure da terminale digitare:
	\begin{lstlisting}
		git clone https://github.com/Mattens15/Inline-LASESI
	\end{lstlisting}
	
	Adesso, in ordine, scrivere i seguenti comandi dal terminale per installare le gemme e avviare il server di rails.
	\begin{lstlisting}
		cd Inline-LASESI
		bin/bundle install
		bin/rake db:create
		bin/rails s
	\end{lstlisting}
	
	\section[Attrvaverso unit e integration testing]{Testing}
	Il testing è una pratica fondamentale che si svolge dopo lo stesura del codice. Serve a prevenire e correggere importanti bug che minano la sicurezza dell'applicativo.
	Ci affidiamo a due strumenti forniti dalla gemme di RoR, \textbf{Rspec} e \textbf{Cucumber}.
	
	Il periodo di testing è durato all'incirca 1 settimana, attraverso il quale sono stati riscontrati errori di sicurezza nella validazione dell'utente. Una qualsiasi persona, con una POST poteva tranquillamente validare un utente.
	
	\subsection{Unit test}
	In \textit{Ingegneria del software}, per unit test si intede l'attività di testing di singole unità di software. Per unità di intende normalmente il minimo componente di un programma dotato di funzionamento autonomo.
	
	In Inline i test di unità sono stati svolti con \textbf{Rspec} e si trovano in \texttt{./spec/}. Quindi ogni controller, model e view è stata testata con Unit test.
	
	
	Prendiamo per esempio il codice di test del controller powers
	
	 \texttt{file ./spec/controller/powers\_controller\_spec.rb}
	\begin{lstlisting}
		RSpec.describe PowersController, type: :controller do
			before(:each) do
				@owner = FactoryBot.create(:owner)
				@user = FactoryBot.create(:user)
				@room = @owner.rooms.create!(attributes_for(:valid_room))
			end
			
			describe "POST #create as room host" do
				it "returns http success" do
					allow(controller).to receive(:current_user).and_return(@owner)
			
					expect{post :create, 
						   params: {room_id: @room.id, power: {user_id: @user.username}}}.to change {@user.powers.count}.by(1)
					expect(response).to redirect_to(edit_room_path(@room.hash_id))
				end
			end
			
			//Other tests
		end
	\end{lstlisting}
	
	Dalla prima riga, capiamo che si sta testando un controller, in particolare PowersController.
	
	\texttt{before(:each) do} implica che prima di ogni test descritto all'interno di questa funziona crei prima un propretario, un utente e una stanza.
	
	Nel test soprastante stiamo testando il fatto che un utente che abbia creato la stanza in precedenza abbia la possibilità di aggiungere altri admin. Abbiamo supposto che il current\_user fosse il proprietario.
	
	Nell' expectation abbiamo effettuato una post, nei parametri è presente l'id della room e l'id dell'utente che vogliamo far diventare admin.
	
	Alla fine ci aspettiamo che il controller reindirizza nel percordo edit\_room\_path.
	
	\subsection{Integration test}
	
	I test di integrazione sono stati svolti con Cucumber, un tool che permette di effettuare BDD, \textit{Behavior driven development}. Questi tests permettono di verificare l'integratezza e l'insieme delle parti, testate nel test di unità, quando devono compiere un'operazione. Queste operazioni sono per noi le \textit{User stories} descritte nel capitolo 2.
			
	I test descritti da cucumber si trovano in \texttt{./features}. La peculiarità di questo strumento è quella di scrivere test in linguaggio di alto livello, quasi parlato; Successivamente, un modulo in linguaggio ruby preleverà gli step con delle espressioni regolari e effettuare dei test.
	
	Prendiamo per esempio il test della user story 18.
	\\
	\texttt{As a User I want to use Mapbox so that i can see rooms in a certain location.}\\
	\texttt{file PASSED\_find\_room\_by\_location.feature}
	\begin{lstlisting}
		@javascript
		Scenario:
			Given I am a registered user
			And I log in
			When I create a room
			And I visit dashboard
			And I fill Search with Via del babuino
			And I pick the first one
			Then I should see marker
	\end{lstlisting}
	
	\texttt{@Javascript} indica al motore che esegue i test che avrà bisogno dell'appoggio di Javascript nel browser per far sì che i test vengano superati.
	
	\textit{Given}, \textit{When} e \textit{Then} rappresentano le parole chiave per i test. In ordine raffigurano il preambolo dello scenario, poi l'azione e alla fine la conseguenza.
	
	In questo caso: \textbf{Dato} che sono registrato e loggato, \textbf{quando} creo una stanza e visito la dashboard e scrivo \textit{Via del Babuino} nella barra di ricerca,\textbf{ dovrei} vedere un marker sulla mappa.